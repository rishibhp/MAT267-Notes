\section{Techniques for solving ODES}

\subsection{Separation of variables} \label{sec:sep-of-var}
Suppose we have a differential equation of the following form
$$ \frac{dx}{dt} = f(t) g(x) $$
where $f, g$ are continuous. Suppose we are also given that $x(0) = x_0$ as our initial condition.

If $g(x_0) = 0$ then $x(t) \equiv x_0$ is a solution. So suppose that $g(x_0) \neq 0$, implying that $g$ is non-zero in some neighbourhood around $x_0$. We can thus do the following
\begin{align*}
    \frac{dx}{dt} &= f(t) g(x)\\
    \frac{x'(t)}{g(x(t))} &= f(t)\\
    \int \frac{x'(t)}{g(x(t))} dt &= \int f(t) dt\\
    \int \frac{1}{g(u)} &= \int f(t) dt\\
    G(x) &= F(t) + c
\end{align*}
where in the penultimate line we substitute $u = x(t)$ and in the final line we take $G$ to be the anti-derivative of $\frac{1}{g}$ and $F$ to be the anti-derivative of $f$. Recall we are solving around $(0, x_0)$ and we know that $g$ is non-zero in non-zero in a neighbourhood of it. Thus $G'(x) = \frac{1}{g(x)}$ is also non-zero (either all positive or all non-negative) implying that $G$ is invertible on this neighbourhood. We can thus take $x = G^{-1}(F(t) + c)$. The solution passing through $(t_0, x_0)$ corresponds to $c = G(x_0) - F(t_0)$.

\begin{remark}
The concrete construction of the solution implies that the solution to a differential equation with separable values is unique in a small neighbourhood of $(t_0, x_0)$, given than $g(x_0) \neq 0$.
\end{remark}

The above steps are often written more simply as 
\begin{align*}
    \frac{dx}{dt} &= f(t) g(x)\\
    \frac{1}{g(x)} dx &= f(t) dt\\
    \int \frac{1}{g(x)} dx &= \int f(t) dt\\
    G(x) = F(t) + c
\end{align*}

\subsubsection{Example}
A simple example of using separation of variables is \autoref{eq:eg3} (no more guesswork required!). A more interesting example can be found in \autoref{sec:logistic-eqn}.

\subsection{Homogeneous Functions}
We say a function $F$ on $\R^2$ is homogeneous of degree $\alpha$ (with $\alpha \in \R$) if $F(tx, ty) = t^\alpha F(x, y)$.

Suppose we are given
\begin{equation}
    P(x, y) dx + Q(x, y) dy = 0
\end{equation}
Where $P$ and $Q$ are homogeneous and of the same degree. Then substituting $y = xu$ (giving us $dy = u dx + x du$) or $x = yv$ (giving us $dx = v dy + y dv$) changes our differential equation into one where we can separate variables.

\subsubsection{Example}
Suppose we have the following equation
\begin{equation}\label{eq:hom-eg}
    \underbrace{\left[ x e^{\frac{y}{x}} - y \sin \left(\frac{y}{x} \right) \right]}_{P(x, y)} dx + \underbrace{x \sin\left( \frac{y}{x} \right)}_{Q(x,y)} dy = 0
\end{equation}
It is easy to verify $P$ and $Q$ are homogeneous functions (of degree 2). We substitute $y = xu$ (either substitution works but often one is easier that the other. In this case it seems quite apparent that one would want to replace $\frac{y}{x}$). The substitution gives us
\begin{align*}
    [xe^u - xu \sin(u)] dx + x \sin(u)(u dx + x du) &= 0\\
    e^u dx + x \sin(u) du &= 0
\end{align*}
(note we divide by $x$ without worry since the original equation \autoref{eq:hom-eg} doesn't allow for $x = 0$)
The variables can now clearly be separated.

\subsection{Exact Differential Equations}\label{sec:exact-diff-eq}
Suppose we have an ODE of the form
$$ P(x, y)dx + Q(x, y)dy = 0 $$
is exact if there exists a function $f(x, y)$ such that
$$ \frac{\partial f}{\partial x} = P(x, y), \frac{\partial f}{\partial y} = Q(x, y) $$
The general solution to such ODEs is given by the one-parameter family
$$ f(x, y) = c $$

It would be nice to know when an ODE is exact. Luckily we have the following theorem which not only tells us when an equation is exact but even gives a construction for the function $f$.
\begin{theorem}
Suppose we have an ODE of the form
$$ P(x, y)dx + Q(x, y)dy = 0 $$
where $P, Q$ are defined on a simply connected region and are $C^1$. Then the equation is exact if and only if
$$ \frac{\partial P}{\partial y} = \frac{\partial Q}{\partial x} $$
\end{theorem}
\begin{proof}
The forward direction follows from a theorem in analysis. Namely if a function $f$ is smooth (or even just $C^1$) then
\begin{align*}
    \frac{\partial^2 f}{\partial x \partial y} = \frac{\partial^2 f}{\partial y \partial x}
\end{align*}

Thus we need show the reverse direction. Thus we assume that
$$ \frac{\partial P}{\partial y} = \frac{\partial Q}{\partial x} $$
Suppose such an $f$ were to exist. Then we know that $f$ must satisfy
\begin{align*}
    \frac{\partial f}{\partial x} = P(x, y) \Rightarrow f(x, y) = \int_{x_0}^{x} P(x, y)dx + R(y)
\end{align*}
Typically we get a constant term when integrating, but in this case the constant term depends on $y$, hence why $R$ becomes a function of $y$. By assumption the derivative of $f$ with respect to $y$ is $Q$. Thus we get that
\begin{align*}
    Q(x, y) = \frac{\partial f}{\partial y} &= \frac{\partial}{\partial y} \left( \int_{x_0}^{x} P(s, y)ds + R(y) \right)\\
    &= \int_{x_0}^{x} \frac{\partial}{\partial y} P(s, y)ds + R'(y)\\
    &= \int_{x_0}^x \frac{\partial }{\partial x} Q(s, y) ds + R'(y)\\
    &= Q(x, y) - Q(x_0, y) + R'(y)
\end{align*}
where of course $x_0$ is some arbitrary constant in the domain. We then conclude that $R'(y) = Q(x_0, y)$. Therefore
$$ R(y) = \int_{y_0}^{y} Q(x_0, s) ds $$
(we ignore the integration constant for now since account for it later when giving the general solution to the ODE). Therefore
$$ f(x, y) = \int_{x_0}^{x} P(s, y)ds + \int_{y_0}^{y} Q(x_0, s)ds $$
All that remains to check is that this $f$ does indeed satisfy the conditions. Now it is clear that
$$ \frac{\partial f}{\partial x} = P(x, y) $$
The other one takes slightly more work
\begin{align*}
    \frac{\partial f}{\partial y} &= \frac{\partial }{\partial y} \int_{x_0}^x P(s, y) ds + Q(x_0, y)\\
    &= \int_{x_0}^x \frac{\partial }{\partial y} P(s, y) ds + Q(x_0, y)\\
    &= \int_{x_0}^{x} \frac{\partial }{\partial x} Q(s, y) ds + Q(x_0, y)\\
    &= Q(x, y) - Q(x_0, y) + Q(x_0, y)\\
    &= Q(x, y)
\end{align*}
\end{proof}

\subsubsection{Example}
Suppose we have the equation
\begin{equation}\label{eq:exact-eg}
    (2x + y \cos x)dx + (2y + \sin x - \sin y)dy = 0
\end{equation}
In this case we have $P(x, y) = 2x + y \cos x$ and $Q(x, y) = 2y + \sin x - \sin y$. Since
$$ \frac{\partial P}{\partial y} = \cos x = \frac{\partial Q}{\partial x} $$
we can conclude using the previous theorem that the differential equation is exact. Thus we can construct $f$ as given by the proof as well. We start with
$$ f(x, y) = \int P(x, y) dx + R(y) = x^2 + y \sin x + R(y) $$
Then we know that
\begin{align*}
    \frac{\partial f}{\partial y} &= Q(x, y)\\
    \sin x + R'(y) = 2y + \sin x - \sin y
\end{align*}
Therefore
$$R(y) = y^2 + \cos y$$
Finally we conclude that the general solution to \autoref{eq:exact-eg} is 
$$ f(x, y) = x^2 + y \sin x + y^2 + \cos y = c $$

\subsubsection{Integrating Factors}
Sometimes an equation is not exact, but we can make it exact by multiplying it with a function. Such a function is called the integration factor. As an example consider the equation
\begin{equation}
    (t^2 x - t) dx + x dt = 0
\end{equation}
Suppose there was some function $h(t)$ (we assume $h$ is a function of $t$ because we know this works. In general it's difficult to know what the integration factor should be a function of). Then we would have
$$ (h(t)(t^2 x - t))dx + (x h(t)) dt = 0 $$
which we assume to be exact. This means that
\begin{align*}
    \frac{\partial }{\partial t} (h(t)(t^2 x - t)) &= \frac{\partial}{\partial x} (x h(t))\\
    h'(t) (t^2 x - t) + h(t) (2tx - 1) &= h(t)\\
    \frac{h'(t)}{h(t)} &= \frac{2 - 2tx}{t^2 x - t} \\
    &= -\frac{2}{t}
\end{align*}
Since the right hand side is a function of $t$, we can integrate to conclude that
$$ \log(h(t)) = -2 \log t $$
or in other words that
$$ h(t) = \frac{1}{t^2} $$

Similar manipulations can be performed if $h$ is function of $x, xt, \frac{x}{t}, \frac{t}{x},$ etc. Often it is not obvious what $h$ should be a function and requires some trial and error. Hence why although we know that any ODE $x' = f(x, t)$ with $f \in C^1$ in a neighbourhood of $(x_0, t_0)$ admits an integration factor, in principle this is a useless technique because it is incredibly hard to find the integration factor.\\

However there do exist some special cases where integration factors \textit{can} be found and these serve as useful examples. Suppose we have an equation of the form
\begin{equation}\label{eq:lin-order-1}
    \frac{dy}{dx} + P(x)y = Q(x)
\end{equation}
Since the derivative and $y$ are raised to the power of $1$, we call this a linear differential equation and specifically a linear differential equation of order 1. One way of solving this equation is to solve the homogeneous version (i.e. set $Q = 0$) using separation of variables and then solve the ODE using variation of constants. A second method, however, is to use integration factors. 

We can first rewrite the equation to get
$$ [P(x) y - Q(x)] dx + dy = 0 $$
Suppose we had an an integrating factor $u(x)$ (in this case we know that the integrating factor is always a function of the dependent variable, in this case $x$). Then we would have
\begin{align*}
    \frac{\partial}{\partial y} [u(x) P(x) y - u(x)Q(x)] &= \frac{\partial}{\partial x} u(x)\\
    u(x) P(x) &= u'(x)\\
    \frac{u'(x)}{u(x)} &= P(x)\\
    u(x) = e^{\int P(x) dx}
\end{align*}
Note that when integrating the exponent we don't need to worry about the integration constant since we don't need the general solution. Substituting this integration factor into \autoref{eq:lin-order-1}  we get
\begin{align*}
    e^{\int P(x) dx} \frac{dy}{dx} + P(x) e^{\int P(x) dx} y &= e^{\int P(x) dx} Q(x)\\
    \frac{d}{dx} (y e^{\int P(x) dx}) &=  e^{\int P(x) dx} Q(x)\\
    y &= e^{- \int P(x) dx} \left( \int e^{\int P(x) dx} Q(x) \right) + c e^{-\int P(x) dx}
\end{align*}
We apologise for the horror we have bestowed upon the reader.


